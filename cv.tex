\documentclass[margin,line]{res}
\usepackage[spanish]{babel}
\usepackage[utf8]{inputenc}
\usepackage{hyperref}

\oddsidemargin -.5in
\evensidemargin -.5in
\textwidth=5.8in
\itemsep=0in
\parsep=0in

\newenvironment{list1}{
  \begin{list}{\ding{113}}{%
      \setlength{\itemsep}{0in}
      \setlength{\parsep}{0in} \setlength{\parskip}{0in}
      \setlength{\topsep}{0in} \setlength{\partopsep}{0in} 
      \setlength{\leftmargin}{0.17in}}}{\end{list}}
\newenvironment{list2}{
  \begin{list}{$\bullet$}{%
      \setlength{\itemsep}{0in}
      \setlength{\parsep}{0in} \setlength{\parskip}{0in}
      \setlength{\topsep}{0in} \setlength{\partopsep}{0in} 
      \setlength{\leftmargin}{0.2in}}}{\end{list}}


\begin{document}

\name{CV - 2021 - Dr. Ing. Ramiro M. Irastorza \vspace*{.1in}}

\begin{resume}
\section{\sc Información de Contacto Laboral}
\vspace{.05in}
\begin{tabular}{@{}p{3.5in}p{3in}}
Calle 59 N$^o789$             & {\it Tel:} +54-221-4233283 \\            
Instituto de Física de Líquidos y Sistemas Biológicos & rirastorza@iflysib.unlp.edu.ar\\         
IFLySiB-CONICET-UNLP & http://www.iflysib.unlp.edu.ar/ \\       
La Plata, PA CC 565 B1900BTE, Argentina. & \\     
\end{tabular}

% \section{\sc 1-Formación académica}
% 
% Investigador Asistente, IFLySiB, CONICET.

\section{\sc 1-Formación académica}
{\bf Facultad de Ingeniería (FI), Universidad Nacional de La Plata (UNLP)}, Argentina\\
%{\em Department of Statistics} 
\vspace*{.1in}
\begin{list1}
\item[] \textbf{Doctorado en Ingeniería Electrónica}, Septiembre de 2010.
\begin{list2}
\vspace*{.1in}
\item Título de tesis:  ``Identificación de sistemas. Aplicación a la evaluación in vitro de la calidad ósea en tejido trabecular humano.'' 
\item Director:  Fernando Vericat
\end{list2}
\vspace*{.3in}
\item[] \textbf{Ingeniero en Electrónica}, Febrero de 2005.
\end{list1}

\section{\sc 2 Docencia Universitaria e Investigación}
\vspace*{-.2in}
\subsection{\sc 2.1 Cargos actualidad}

{\bf Investigador Adjunto} \hfill {\bf Diciembre de 2019  - presente}\\
Instituto de Física de Líquidos y Sistemas Biológicos.\\
{\em CONICET CCT La Plata}, Argentina.\\

{\bf Profesor Titular Interino} \hfill {\bf Mayo de 2019  - presente}\\
Introducción a los Elementos Finitos, Facultad de Ingeniería.\\
{\em Facultad Regional La Plata - Universidad Tecnológica Nacional (FRLP-UTN)}, Argentina\\

{\bf Profesor Adjunto Ordinario} \hfill {\bf Marzo de 2020  - presente}\\
Cursos de grado de Bioinformática, Instituto de Ingeniería y Agronomía.\\
{\em Universidad Nacional Arturo Jauretche (UNAJ)}, Argentina\\

{\bf Profesor Ajunto Ordinario} \hfill {\bf Enero de 2010  - (actualmente de licencia)}\\
Cursos de grado de tecnología y control para Diseño Industrial, Facultad de Bellas Artes (FBA).\\
{\em Universidad Nacional de La Plata (UNLP)}, Argentina
\vspace*{-.1in}
\subsection{\sc 2.2 Cargos anteriores}
\vspace*{-.1in}
{\bf Profesor Titular Interino}, Tecnología para Diseño Industrial, FBA-UNLP, de 2015 a 2017.

{\bf Profesor Contratado}, Física I, Instituto de Ingeniería y Agronomía, UNAJ, de 2013 a 2017.

{\bf Profesor Adjunto Interino por concurso}, Estadística, FI-UNLP, de 2013 a 2015.

{\bf Profesor Adjunto Ordinario}, Tecnología para Diseño Industrial, FBA-UNLP, de 2009 a 2015.

{\bf Jefe de Trabajos Prácticos Ordinario}, Probabilidades, FI-UNLP, de 2006 a 2015.

{\bf Ayudante Diplomado Ordinario}, Probabilidades y Estadística, FI-UNLP, de 2004 a 2006.

{\bf Ayudante Alumno}, Dispositivos Electrónicos, FI-UNLP. de 2000 a 2004.

{\bf Ayudante Alumno}, Probabilidades y Estadística, FI-UNLP, de 2000 a 2004.

\section{\sc 3 Actividad y Producción en Docencia Universitaria}
\vspace*{-.2in}

\subsection{\sc 3.1 Innovación pedagógica}

\textbf{Curso de posgrado virtual Complementos de Matemática Aplicada} en la UNAJ. Material de estudio en plataforma Moodle (se implementaron foros, cuestionarios, software para ejemplos, y notas de la materia). Disponible en: \url{https://campus.unaj.edu.ar/}.

\textbf{Curso de posgrado virtual Métodos Estadísticos} en la UNAJ. Material de estudio en plataforma Moodle (se implementaron foros, cuestionarios, software para ejemplos, y notas de la materia). Disponible en: \url{https://campus.unaj.edu.ar/}.

% {\bf Consejo Directivo} IFLySiB CCT La Plata CONICET. Representante de becarios. Período 2011--2014.
% 
% {\bf Comisión evaluadora de Personal de Apoyo} de CONICET de IFLySiB CCT La Plata CONICET. Período 2011.

\subsection{\sc 3.2 Material didáctico}

\textbf{Notas de clase y ejemplos de código.} Material de estudio para la materia Introducción a los Elementos Finitos. Disponible en: \url{https://github.com/rirastorza/Intro2FEM}.


\textbf{Notas de clase.} Material de estudio para la materia Tecnología IV para Diseño Industrial. Tema: Introducción a los materiales: propiedades y dimensionamiento. \footnote{Se creo el \href{https://sites.google.com/site/catedratecnologia/}{sitio de google}. En la actualidad como estoy de licencia en este cargo se puede acceder al material mediante este \href{https://docs.google.com/viewer?a=v\&pid=sites\&srcid=ZGVmYXVsdGRvbWFpbnxjYXRlZHJhdGVjbm9sb2dpYXxneDoyMmYyZmRiNWYzN2JkNDMy}{link}.}

\subsection{\sc 3.3 Docencia en posgrado acreditada}

{\bf Análisis estadístico utilizando R}, Universidad Nacional de Quilmes. Coordinador: Dr. Marcelo Cappelletti y Ramiro M. Irastorza. Duración: 36 horas. Julio a septiembre de 2020. Resolución RES.(R) No 1166/19.

{\bf Radiofrecuencia y tejidos biológicos}, Departamento de Ingeniería Electrónica, Universidad Politécnica de Valencia, España. Duración: 20 horas. Docente: Ramiro M. Irastorza. En 2017.

{\bf Métodos Estadísticos}, Instituto de Ingeniería y Agronomía, UNAJ. Coordinador: Dr. Marcelo Cappelletti y Ramiro M. Irastorza. Duración: 30 horas.  En la actualidad. Resolución CS Número: 002--15 Expt. Número: 2100/14.

{\bf Complementos de Matemática Aplicada}, Instituto de Ingeniería y Agronomía, UNAJ. Coordinador: Dr. Marcelo Cappelletti y Ramiro M. Irastorza. Duración: 40 horas.  Desde 2016 a la actualidad. Resolución CS Número: 003--15 Expt. Número: 2101/14.

{\bf Herramientas computacionales para científicos}, Docente invitado, Instituto de Física de Líquidos y Sistemas Biológicos, Facultades de Ciencias Exactas y de Ciencias Astronómicas y Geofísicas de la UNLP, y Facultad Regional La Plata de la UTN. Coordinadores: Dr. Luis Pugnaloni y Dr. Manuel Carlevaro. Duración: 70 horas. Desde 2008 a la actualidad. Expediente: 1100--2822/17 en Facultad de Ciencias Astronómicas y Geofísicas de UNLP. Ordenanza: 1478 de Ministerio de Educación Universidad Tecnológica Nacional Rectorado. Expediente: 0700--008592/16--000 en Facultad de Ciencias Exactas de UNLP.


\subsection{\sc 3.4 Integrante de jurados de concurso docente}

Jurado en concurso por el cargo de {\bf Profesor Titular Ordinario con Dedicación Exclusiva} para el dictado de un curso y realizar tareas de coordinación de las Cátedras  F307--Estadística y F312--Probabilidades del Área Matemática Aplicada del Departamento de Ciencias Básicas, FI-UNLP, que se tramitó por {\bf Expediente Número 0300-11750/13}.

Jurado en concurso por el cargo de {\bf Jefe de Trabajos Prácticos Ordinario con Dedicación semi exclusiva  a la docencia} para cumplir tareas en las asignaturas F307--Estadística y F312--Probabilidades del Área Matemática Aplicada del Departamento de Ciencias Básicas de la FI-UNLP, que se tramitó por {\bf Expediente Número 0300-001267/14-000}.


 \subsection{\sc 3.5 Integrante de tribunal de tesis de doctorado}

{\bf Tesista: Patricio Gross} dirigida por el Dr. Juan Bava. Facultad de Ingeniería Universidad Nacional de La Plata. 28 de noviembre de 2019. 
 
{\bf Tesista: Ana González Suárez} dirigida por el Dr. Enrique Berjano. Universidad Politécnica de Valencia. 24 de febrero de 2014.

\subsection{\sc 3.6 Integrante de tribunal de proyecto final de carrera}

{\bf Práctica Profesional Supervisada (PPS) de Mauro Salina} dirigido por Dr. Jorge Osio. Universidad Nacional Arturo Jauretche. 25 de febrero de 2021.

{\bf Trabajo Final de Grado de Agustina Lago} dirigido por Dr. Andrés Salvay y Dra. Mercedes Peltzer. Universidad Nacional de Quilmes. 12 de noviembre de 2020.

{\bf Práctica Profesional Supervisada (PPS) de Leandro Jesús Charlier} dirigido por Dr. Marcelo Cappelletti y Dr. Waldo Hasperué. Universidad Nacional Arturo Jauretche. 13 de julio de 2017.

\section{\sc 4 Formación de recursos humanos en Investigación, Desarrollo e Innovación}
\vspace*{-.2in}
\subsection{\sc 4.1 Dirección o codirección de tesis de doctorado}

{\bf Dirección de Ing. María José Cervantes}. Estudiante de doctorado de Facultad de Ciencias Exactas, Universidad Nacional de La Plata. Fecha estimada de finalización en 2024.

{\bf Codirección de Dr. Jesús E. Fajardo} dirigido por Dr. Fernando Vericat. Doctorado de Facultad de Ciencias Exactas, Universidad Nacional de La Plata. Fecha estimada de finalización: febrero de 2020.

\subsection{\sc 4.2 Dirección de proyecto final de grado}

{\bf Codirección de Jesús Serrano} dirigido por Profesora Eugenia Blangino. Año de defensa 2011. Departamento de Ingeniería Mecánica, Facultad de Ingeniería, Universidad de Buenos Aires.

\subsection{ \sc 4.3 Dirección o codirección de investigadores}
  
Hugo Ariel Álvarez. Docente Investigador UNLP. \textbf{Director} (Proyecto UNAJ Investiga 2017, (2018-2019)).
  
C. Manuel Carlevaro. Investigador Independiente CONICET. \textbf{Director} (Proyecto PICT 2016-2303, (2018-2020)).

Federico Lotto. Becario Posdoctoral CONICET. \textbf{Director} (Proyecto PICT 2016-2303, (2018-2020)).
  
Ariel G. Meyra. Investigador Adjunto CONICET. \textbf{Director} (Proyecto PICT 2016-2303, (2018-2020)).

Fernando Vericat. Investigador Principal Jubilado. \textbf{Director} (Proyecto PICT 2016-2303, (2018-2020)).

Guillermo Zarragoicoechea. Investigador Independiente CIC--PBA. \textbf{Director} (Proyecto PICT 2016-2303, (2018-2020)).

\subsection{\sc 4.4 Dirección de becarios de grado}

{\bf Dirección de beca Estímulo a las Vocaciones Científica 2020}. Estudiante: Julián Gómez, Universidad Nacional Arturo Jauretche.

{\bf Dirección de beca Estímulo a las Vocaciones Científica 2018}. Estudiante: Florencia Ayala, Universidad Nacional Arturo Jauretche.

{\bf Dirección de beca Estímulo a las Vocaciones Científica 2015}. Estudiante: Luciano Olmedo, Universidad Nacional Arturo Jauretche.

\subsection{\sc 4.5 Dirección de auxiliares de docencia}

Durante mis periodos de {\bf Profesor Adjunto y Titular} tanto en FBA-UNLP como en FI-UNLP entre 2009 a 2017 tuve a cargo Profesores Adjuntos, Jefes de Trabajos Prácticos, Ayudantes Diplomados, y Ayudantes Alumnos.

\section{\sc 5 Actividad en investigación, desarrollo e Innovación}
\vspace*{-.2in}

\subsection{\sc 5.1 Director de proyecto homologado}

\textbf{Director de ``Desarrollo e implementación de una metodología para la evaluación \textit{in vivo} de la calidad ósea''}. Otorgado por la Agencia Nacional de Promoción Científica y Tecnológica, PICT 2016-2303, (2018--2020).

\textbf{Director de ``Resolución computacional de problemas térmicos y electromagnéticos en aplicaciones biomédicas''}. Proyecto UNAJ Investiga 2017 código: 80020170100019UJ, otorgado por la Universidad Nacional Arturo Jauretche, (2019--2020).

\textbf{Director de ``Simulación y evaluación de propiedades dieléctricas de tejidos óseos''}. Proyecto de Investigación Científica y Tecnológica (PICT-2012 Jóvenes). Otorgado por Ministerio de Ciencia, Tecnología e Innovación Productiva / Agencia Nacional de promoción Científica y Tecnológica, (2013--2015).

\textbf{Subsidios para Jovenes Investigadores 2011}. Otorgado por Universidad Nacional de La Plata, (2011--2012).

\subsection{\sc 5.2 Codirector de proyecto homologado}

\textbf{``Utilización de métodos numéricos y de técnicas subsimbólicas de la inteligencia artificial para la generación de energías  limpias y la evaluación de la salud ósea''}. Proyecto de Investigación Orientado (PIO-00001). Participan: Universidad Nacional Arturo Jauretche, IFLySiB. Otorgado por CONICET, (2016-2018).

\subsection{\sc 5.3 Evaluación de proyectos científicos, tecnológicos e ingeniería}

{\bf Convocatoria de Proyectos: OCS 136/08}. Evaluador externo de la Universidad Nacional de Mar del Plata durante 2020.

{\bf Convocatoria PICT-2017}. Evaluador de Proyectos de Investigación Científica y Tecnológica (2017) Plan Argentina Innovadora 2020 - Jóvenes, integrando el Banco de evaluadores del FONCyT, 11 de setiembre de 2017.

{\bf Proyecto de investigación Fundamental Fondo Clemente Estable 2017}. Agencia Nacional de Investigación e Innovación (Uruguay).

{\bf Convocatoria PICTO-UNLaM-2013}. Evaluador de proyecto, integrando el Banco de evaluadores del FONCyT, 26 de noviembre de 2014.

\subsection{\sc 5.4 Integrante de proyecto con cuatro años o más de antigüedad}

\textbf{Estudio de fluidos confinados en sistemas de interés tecnológico}. Proyectos de Investigación de Unidades Ejecutoras 2018, en ejecución otorgado por CONICET, Titular: Dr. Santiago Grigera. Responsable científico: Dr. Manuel Carlevaro, (2019--2023). 

\textbf{Termodinámica Estadística}. Otorgado por la Universidad Nacional de La Plata, Proyecto 11/I108. Titular: Fernando Vericat.(2006--2009). 

\subsection{\sc 5.5 Integrante de proyecto}

\textbf{Tecnologías basadas en energía de radiofrecuencia y microondas para cirugía de mínima invasión}. Participan: Universidad Politécnica de Valencia, IFLySiB. Otorgado por Ministerio de Economía y Competitividad, España, (2015--2017).

\textbf{Líquidos clásicos y fermiónicos: Estudio teórico y computacional}. Proyecto plurianual otorgado por el CONICET, PIP 112-201201-00154. Titular: Fernando Vericat. (2014--2016).

\textbf{Termodinámica estadística}. Otorgado por la Universidad Nacional de La Plata. Proyecto 11/I153. Titular: Fernando Vericat. (2010--2012). 

\textbf{Mise en place d’un laboratoire international pour la recherche sur les matériaux structures recouvrements pour applications médicales}. Participan: Université Laval-INTI-Laboratorio de Biomecánica FI-UBA. Otorgado por Agence Universitaire de la Francofonie (AUF), (2010--2012).

\textbf{Estudio teórico y computacional de líquidos.} Proyecto plurianual otorgado por el CONICET, PIP  112-200801-01192. Titular: Fernando Vericat. (2009--2011).

\textbf{Propiedades termodinámicas, estructurales y electrónicas de líquidos. Teoría y simulación.} Otorgado por la Agencia Nacional de Promoción Científica y Tecnológica, PICT 2007-00908. Titular: Fernando Vericat. (2008--2009).

\textbf{Teoría y Simulación de Líquidos.} Proyecto plurianual otorgado por el CONICET, PIP Nro. 6240. Titular: Fernando Vericat. (2006--2008).\\


\section{\sc 6 Producción en investigación, desarrollo, innovación y transferencia}
\vspace*{-.2in}
\subsection{\sc 6.1 Publicación con referato en revistas internacionales}


Fajardo J., Lotto F.P., Vericat F., Carlevaro C.M., \textbf{Irastorza R.M.}, ``Microwave tomography with phaseless data on the calcaneus by means of artificial neural networks'', Medical \& biological engineering \& computing (2020).

\textbf{Irastorza R.M.}, Gonzalez-Suarez A., Pérez J.J., Berjano E., ``Differences in applied electrical power between full thorax models and limited-domain models for RF cardiac ablation'', International Journal of Hyperthermia (2020).

Fajardo J.E., Galván J., Vericat F., Carlevaro C.M., \textbf{Irastorza R.M.}, ``Phaseless Microwave Imaging of Dielectric Cylinders: an Artificial Neural Networks-Based Approach'', Progress In Electromagnetics Research (2019).

Fajardo J., Vericat F., Irastorza G., Carlevaro C.M., \textbf{Irastorza R.M.},
``Sensitivity analysis on imaging the calcaneus using microwaves'', Biomedical Physics \& Engineering Express, (2019).

Sánchez H.R., {\bf Irastorza R.M.}, Carlevaro C.M., ``Uncertainties and temperature correction in molecular dynamic simulations of dielectric properties of condensed polar systems'', Journal of Molecular Liquids (2019).

Fajardo J.E., Vericat F., Carlevaro C.M., Berjano E., \textbf{Irastorza R.M.}, ``Effect of the trabecular bone microstructure on measuring its thermal conductivity: A computer modeling-based study'', Journal of thermal biology (2018).

\textbf{Irastorza R.M.}, d'Avila A., Berjano E., ``Thermal latency adds to lesion depth after application of high‐power short‐duration radiofrequency energy: Results of a computer‐modeling study'', Journal of cardiovascular electrophysiology (2018).

\textbf{Irastorza R.M.}, Trujillo M., Berjano E., ``How coagulation zone size is underestimated in computer modeling of RF ablation by ignoring the cooling phase just after RF power is switched off'', International journal for numerical methods in biomedical engineering (2017).

\textbf{Irastorza R.M.}, Trujillo M., Martel Villagrán J., Berjano E., ``Computer modelling of RF ablation in cortical osteoid osteoma: Assessment of the insulating effect of the reactive zone'', International Journal of Hyperthermia (2016).

Carlevaro C.M., {\bf Irastorza R.M.}, Vericat F., ``Chirality in a quaternionic representation of the genetic code'', Biosystems, (2016).

Carlevaro C. M., \textbf{Irastorza R. M.}, Vericat F., ``Quaternionic representation of the genetic code'', Biosystems (2016).

\textbf{Irastorza R.M.}, Drouin B., Blangino E., Mantovani D., ``Mathematical Modeling of Uniaxial Mechanical Properties of Collagen Gel Scaffolds for Vascular Tissue Engineering'', The Scientific World Journal (2015).

\textbf{Irastorza R. M.}, Blangino E., Carlevaro C. M., Vericat F., ``Modeling of the dielectric properties of trabecular bone samples at microwave frequency. Medical \& biological engineering \& computing, (2014).

\textbf{Irastorza R.M.}, Carlevaro C.M., Pugnaloni L.A., ``Exact predictions from the Edwards ensemble versus realistic simulations of tapped narrow two-dimensional granular columns'', Journal of Statistical Mechanics: Theory and Experiment, (2013).

\textbf{Irastorza R.M.}, Carlevaro C.M., Verica F., ``Is there any information on micro-structure in microwave tomography of bone tissue?'', Medical Engineering \& Physics (2012).

Brusca M.I., {\bf Irastorza R.M.}, Cattoni D.I., Ozu M., Chara O., ``Mechanisms of interaction between Candida albicans and Streptococcus mutans: An experimental and mathematical modelling study'', Acta Odontologica Scandinavica, (2012).

\textbf{Irastorza R.M.}, Achilli M., Amadei M., Blangino E., Drouin B., Mantovani D., ``Ultrasonic setup for testing hydrogels: preliminary experiments on collagen gels.'', Advanced Materials Research, (2012).

\textbf{Irastorza R.M.}, Mayosky M.A., Grigera J.R., Vericat F., ``Dielectric properties of natural and demineralized collagen bone matrix'', IEEE Trans. on Dielectrics and Electrical Insulation, (2011).

Blangino E., {\bf Irastorza R.M.}, Valente S., Vericat F., ``Experimental Techniques to Evaluate In Vitro Trabecular Bone Properties and Emerging Numerical Model'', Materials Science Forum, (2010).

{\bf Irastorza R.M.}, Valente S., Vericat F., Blangino E., ``Dielectric Properties in Fresh Trabecular Bone Tissue from 1MHz to 1000MHz: A Fast and Non Destructive Quality Evaluation Technique'', Materials Science Forum, (2010).

{\bf Irastorza R.M.}, Mayosky M.A., F Vericat F., ``Noninvasive measurement of dielectric properties in layered structure: A system identification approach'', Measurement, (2009).

\subsection{\sc 6.2 Publicación con referato en revistas nacionales}

Fajardo J.E., Vericat F., Carlevaro C.M., Irastorza R.M., ``Effects of Cancellous Bone Dielectric Variability on Microwaves Detection Feasibility. A Simulation Study'', Revista Argentina de Bioingeniería  (2018). Congreso Argentino de Bioingeniería 2017.

Guzmán J.V., Cappelletti M.A., Irastorza R.M., Morales D.M., ``Taller de actualización de recursos y herramientas TIC aplicadas a la enseñanza de la Física en la escuela media'', XI Congreso de Tecnología en Educación y Educación en Tecnología (TE\&ET 2016).

Cappelletti M.A., Irastorza R.M., Morales D.M., ``Estudio de dispositivos y materiales semiconductores bajo condiciones adversas de funcionamiento a través de herramientas TCAD'', XVI Workshop de Investigadores en Ciencias de la Computación, (2014).

Irastorza R.M., Vericat F., Grigera J.R., ``Mediciones dieléctricas no invasivas en medios estratificados utilizando líneas coaxiales abiertas'', XI Reunión de Procesamiento de Información y Control, Río Cuarto, Argentina, (2005).

\subsection{\sc 6.3 Presentaciones en reuniones científicas}

Irastorza R.M., ``Tomografía de microondas'', Metodologías multiescala de aplicación en Osteología (2020), CABA. Presentación oral en Asociación Argentina de Osteología y Metabolismo Mineral (AAOMM).

Fajardo J.E., Ayala F., Carlevaro C.M., Vericat F., Irastorza R.M., ``Microestructura y propiedades térmicas de hueso trabecular: Mediciones y simulaciones computacionales'', Congreso Anual de la Sociedad Argentina de Física (AFA 2018), CABA. Presentación tipo poster.

Fajardo J.E., Carlevaro C.M., Vericat F., Irastorza R.M., ``Medición mínimamente invasiva de conductividad y difusividad térmica de tejido óseo'', Congreso Anual de la Sociedad Argentina de Física (AFA 2017), La Plata. Presentación tipo poster.

Fajardo J.E., Carlevaro C.M., Vericat F., Irastorza R.M., ``Modelo realista de tomografía de microondas en el calcáneo'', Congreso Anual de la Sociedad Argentina de Física (AFA 2016), Tucumán. Presentación oral y poster.

Irastorza R.M., Trujillo M., Villagrán J.M., Berjano E., ``Radiofrequency Ablation of Osteoma Osteoide: A Finite Element Study'', IFMBE Proceedings, VI Latin American Congress on Biomedical Engineering, 2015, Paraná, Argentina. 

Fajardo J.E., Carlevaro C.M., Irastorza R.M., Vericat F., ``Diseño y construcción de prototipo para la medición de conductividad y difusividad térmica de tumores óseos'', Congreso Anual de la Sociedad Argentina de Física (AFA 2015).

Irastorza R.M., Carlevaro C.M., Vericat F., ``Tomografía de microondas en tejido óseo: un estudio de simulación'', Congreso Anual de la Sociedad Argentina de Física (AFA 2014), Tandil. Presentación en poster.

Drouin B., Gagnon R., Lacombe J., Irastorza R. M., Mantovani D., ``Viscoelastic models of Collagen hydrogels scaffolds used in vascular tissue engineering'', The World Congress of Biomechanic, 2014, Boston, USA.

Irastorza R.M, Blangino E., Drouin B., Mantovani D., ``Non-Invasive Ultrasonic Assessment of the Mechanical Properties of Hydrogels'', 9th World Biomaterials Congress, 2012, Chengdu, China.

Blangino E., Irastorza R.M., Valente S., Vericat F., ``Experimental Techniques to Evaluate In Vitro Trabecular Bone Properties and Emerging Numerical Model'', International Conference on Processing \& Manufacturing of Advanced Materials, 2009 Berlin, Germany.

\subsection{\sc 6.4 Publicación en revistas sin referato}

Fajardo J.E., Galván J., Vericat F., Carlevaro C.M., Irastorza R.M., ``Phaseless microwave imaging of dielectric cylinders: an artificial neural networks-based approach'', enviado arXiv preprint (2019). Enviado para su publicación.

Fajardo J.E., Lotto F.P., Vericat F., Carlevaro C.M., Irastorza R.M., ``Microwave Tomography with phaseless data on the calcaneus by means of artificial neural networks'', arXiv preprint arXiv:1902.07777, (2019). Enviado para su publicación.

\subsection{\sc 6.5 Evaluación de actividades científicas y técnicas}

Desde 2012 Revisor en revistas científicas:\\
\begin{itemize}
 \item International Journal of Hyperthermia
 \item International Journal for Numerical Methods in Biomedical Engineering
 \item Medical Engineering \& Physics
 \item Progress In Electromagnetics Research
\end{itemize}

\subsection{\sc 6.6 Desarrollo tecnológico verificado fehacientemente}

Convenio de cooperación institucional entre el Instituto Argentino de Radioastronomía (IAR) y el Instituto de Física de Líquidos y Sistemas Biológicos (IFLySiB) CCT--CONICET (20/09/2018). El proyecto inicial de la colaboración se titula ``Imágenes de Tomografía por Microondas''. Se está construyendo el dispositivo y desarrollando el software financiado por el proyecto PICT 2016-2303.

\subsection{\sc 6.7 Divulgación científica y/o tecnológica documentada}

Irastorza, R.M., ``Más vale MAV en compu que cien volando'', artículo escrito para CienciaNet \footnote{CienciaNet es un sitio independiente que publica notas breves de divulgación científica desde 2007 prestando especial atención a producciones de científicos en Argentina \url{http://ciencianet.com.ar}.} (2009).

Irastorza, R.M., ``¿Wi Tricity? Transferencia de potencia sin cables'', artículo escrito para CienciaNet (2007).

% \section{\sc Participación en subsidios} 
% 
% 
% 
% Propiedades termodinámicas, estructurales y electrónicas de líquidos. Teoría y simulación. Participan: IFLYSIB CONICET-UNLP. Otorga: Ministerio de Ciencia, Tecnología e Innovación Productiva / Agencia Nacional de promoción Científica y Tecnológica, (2009-2011).
% 
% \section{\sc Subsidios obtenidos} 
% 
% 
% % \end{list2}
% 
% \section{\sc Formación de recursos\\ humanos} 
% 
% Tesista de posgrado: Fajardo, E.J. Co-dirección de Tesis de doctoral: ``Propiedades dieléctricas y térmicas de tejidos óseos. Aplicaciones biofísicas y biomédicas.''. Lugar de realización: IFLySiB (2015-actualidad).

\end{resume}
\end{document}




