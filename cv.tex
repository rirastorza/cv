\documentclass[margin,line]{res}
\usepackage[spanish]{babel}
\usepackage[utf8]{inputenc}
\usepackage{hyperref}

\oddsidemargin -.5in
\evensidemargin -.5in
\textwidth=5.8in
\itemsep=0in
\parsep=0in

\newenvironment{list1}{
  \begin{list}{\ding{113}}{%
      \setlength{\itemsep}{0in}
      \setlength{\parsep}{0in} \setlength{\parskip}{0in}
      \setlength{\topsep}{0in} \setlength{\partopsep}{0in} 
      \setlength{\leftmargin}{0.17in}}}{\end{list}}
\newenvironment{list2}{
  \begin{list}{$\bullet$}{%
      \setlength{\itemsep}{0in}
      \setlength{\parsep}{0in} \setlength{\parskip}{0in}
      \setlength{\topsep}{0in} \setlength{\partopsep}{0in} 
      \setlength{\leftmargin}{0.2in}}}{\end{list}}


\begin{document}

\name{CV - 2025 - Dr. Ing. Ramiro M. Irastorza \vspace*{.1in}}

\begin{resume}
\section{\sc Información de Contacto Laboral}
\vspace{.05in}
\begin{tabular}{@{}p{3.5in}p{3in}}
Calle 59 N$^o789$             & {\it Tel:} +54-221-4233283 \\            
Instituto de Física de Líquidos y Sistemas Biológicos & rirastorza@iflysib.unlp.edu.ar\\         
IFLySiB-CONICET-UNLP & http://www.iflysib.unlp.edu.ar/ \\       
La Plata, PA CC 565 B1900BTE, Argentina. & \\     
\end{tabular}

% \section{\sc 1-Formación académica}
% 
% Investigador Asistente, IFLySiB, CONICET.


\section{\sc Investigación}

{\bf Investigador Adjunto } \hfill {\bf Mayo de 2019  - presente}\\
 Carrera del Investigador Científico y Tecnológico.\\
{\em Consejo Nacional de Investigaciones Científicas y Técnicas (CONICET - La Plata).}\\

\section{\sc 1-Formación académica}
{\bf Facultad de Ingeniería (FI), Universidad Nacional de La Plata (UNLP)}, Argentina\\
%{\em Department of Statistics} 
\vspace*{.1in}
\begin{list1}
\item[] \textbf{Doctorado en Ingeniería Electrónica}, Septiembre de 2010.
\begin{list2}
\vspace*{.1in}
\item Título de tesis:  ``Identificación de sistemas. Aplicación a la evaluación in vitro de la calidad ósea en tejido trabecular humano.'' 
\item Director:  Fernando Vericat
\end{list2}
\vspace*{.3in}
\item[] \textbf{Ingeniero en Electrónica}, Febrero de 2005.
\end{list1}

\section{\sc 2 Docencia Universitaria}
\vspace*{-.2in}
\subsection{\sc 2.1 Cargos actualidad}

{\bf Profesor Titular Interino} \hfill {\bf Mayo de 2019  - presente}\\
Introducción a los Elementos Finitos, Facultad de Ingeniería.\\
{\em Facultad Regional La Plata - Universidad Tecnológica Nacional (FRLP-UTN)}, Argentina.

\vspace*{-.1in}
\subsection{\sc 2.2 Cargos anteriores}
\vspace*{-.1in}

{\bf Profesor Adjunto Ordinario}, Conceptos de Bioinformática, Instituto de Ingeniería y Agronomía, Universidad Nacional Arturo Jauretche (UNAJ),  {\bf Febrero de 2017  - licenciado}.

{\bf Profesor Titular Interino}, Tecnología para Diseño Industrial, FBA-UNLP, de 2015 a 2017.

{\bf Profesor Contratado}, Física I, Instituto de Ingeniería y Agronomía, UNAJ, de 2013 a 2017.

{\bf Profesor Adjunto Interino por concurso}, Estadística, FI-UNLP, de 2013 a 2015.

{\bf Profesor Adjunto Ordinario}, Tecnología para Diseño Industrial, FBA-UNLP, de 2009 a 2019.

{\bf Jefe de Trabajos Prácticos Ordinario}, Probabilidades, FI-UNLP, de 2006 a 2015.

{\bf Ayudante Diplomado Ordinario}, Probabilidades y Estadística, FI-UNLP, de 2004 a 2006.

{\bf Ayudante Alumno}, Dispositivos Electrónicos, FI-UNLP. de 2000 a 2004.

{\bf Ayudante Alumno}, Probabilidades y Estadística, FI-UNLP, de 2000 a 2004.

\vspace*{.3in}
\section{\sc 3 Actividad y Producción en Docencia Universitaria}
\vspace*{-.1in}

\subsection{\sc 3.1 Innovación pedagógica}

\textbf{Curso de posgrado virtual Método de Elementos Finitos con software libre} en la UTN-FRLP. Material de estudio en plataforma Moodle (se implementaron foros, cuestionarios, software para ejemplos, y notas de la materia), videos en Youtube (diponible en @ramiro\_irastorza). Material de clase en {\it jupyter notebooks} disponible en: \url{https://github.com/rirastorza/Intro2FEM} y en cvg de UTN-FRLP.

\textbf{Curso de posgrado virtual Complementos de Matemática Aplicada} en la UNAJ. Material de estudio en plataforma Moodle (se implementaron foros, cuestionarios, software para ejemplos, y notas de la materia). Disponible en: \url{https://campus.unaj.edu.ar/}.

\textbf{Curso de posgrado virtual Métodos Estadísticos en Diplomatura en Ciencias de Datos} en la UNAJ. Material de estudio en plataforma Moodle (se implementaron foros, cuestionarios, software para ejemplos, y notas de la materia). Disponible en: \url{https://campus.unaj.edu.ar/}.

%{\bf Consejo Directivo} IFLySiB CCT La Plata CONICET. Representante de becarios. Período 2011--2014.

%{\bf Comisión evaluadora de Personal de Apoyo} de CONICET de IFLySiB CCT La Plata CONICET. Período 2011.

\subsection{\sc 3.2 Material didáctico}

\textbf{Notas de clase.} Material de estudio para la asignatura de grado Introducción a los Elementos Finitos en {\it jupyter notebooks} y presentaciones disponible en: \url{https://github.com/rirastorza/Intro2FEM} y en campus virtual de UTN-FRLP.

\textbf{Notas de clase.} Material de estudio para el curso de posgrado Herramientas Computacionales para Científicos presentaciones y ejemplos de código disponible en el curso de Moodle: \url{https://cursos.iflysib.unlp.edu.ar/}.
%
% \textbf{Notas de clase.} Material de estudio para la materia Tecnología IV para Diseño Industrial. Tema: Introducción a los materiales: propiedades y dimensionamiento. \footnote{Se creo el \href{https://sites.google.com/site/catedratecnologia/}{sitio de google}. En la actualidad ya no estoy de docente en ese curso \href{https://docs.google.com/viewer?a=v\&pid=sites\&srcid=ZGVmYXVsdGRvbWFpbnxjYXRlZHJhdGVjbm9sb2dpYXxneDoyMmYyZmRiNWYzN2JkNDMy}{link}.}

\subsection{\sc 3.3 Docencia en posgrado acreditada}

{\bf Método de Elementos Finitos con software libre}, Facultad Regional La Plata de la UTN Cursos de Actualización de Posgrado del Doctorado en Ingeniería, Mención Materiales. Coordinador: Dr. Ariel G. Meyra y Ramiro M. Irastorza. Duración: 100 horas. Desde el 2020 a la actualidad. Resolución CD Número: 77/20. Ordenanza Número: 1776.

{\bf Radiofrecuencia y tejidos biológicos}, Departamento de Ingeniería Electrónica, Universidad Politécnica de Valencia, España. Duración: 20 horas. Docente: Ramiro M. Irastorza. En 2017.

{\bf Métodos Estadísticos}, Instituto de Ingeniería y Agronomía, UNAJ. Coordinador: Dr. Marcelo Cappelletti y Ramiro M. Irastorza. Duración: 30 horas.  En la actualidad. Resolución CS Número: 002--15 Expt. Número: 2100/14.

{\bf Complementos de Matemática Aplicada}, Instituto de Ingeniería y Agronomía, UNAJ. Coordinador: Dr. Marcelo Cappelletti y Ramiro M. Irastorza. Duración: 40 horas.  Desde 2016 a la actualidad. Resolución CS Número: 003--15 Expt. Número: 2101/14.

{\bf Herramientas computacionales para científicos}, Instituto de Física de Líquidos y Sistemas Biológicos, Facultades de Ciencias Exactas y de Ciencias Astronómicas y Geofísicas de la UNLP, y Facultad Regional La Plata de la UTN. Coordinadores: Dr. Manuel Carlevaro y Ramiro M. Irastorza. Duración: 70 horas. Desde 2008 a la actualidad. Expediente: 1100--2822/17 en Facultad de Ciencias Astronómicas y Geofísicas de UNLP. Ordenanza: 1478 de Ministerio de Educación Universidad Tecnológica Nacional Rectorado. Expediente: 0700--008592/16--000 en Facultad de Ciencias Exactas de UNLP.


\subsection{\sc 3.4 Integrante de jurados de concurso docente}

Jurado en concurso por el cargo de {\bf Profesor Titular Ordinario con Dedicación Exclusiva} para el dictado de un curso y realizar tareas de coordinación de las Cátedras  F307--Estadística y F312--Probabilidades del Área Matemática Aplicada del Departamento de Ciencias Básicas, FI-UNLP, que se tramitó por {\bf Expediente Número 0300-11750/13}.

Jurado en concurso por el cargo de {\bf Jefe de Trabajos Prácticos Ordinario con Dedicación semi exclusiva  a la docencia} para cumplir tareas en las asignaturas F307--Estadística y F312--Probabilidades del Área Matemática Aplicada del Departamento de Ciencias Básicas de la FI-UNLP, que se tramitó por {\bf Expediente Número 0300-001267/14-000}.


 \subsection{\sc 3.5 Integrante de tribunal de tesis de doctorado}

{\bf Tesista: Matías Oliva} dirigido por Dr. Ing. Pablo García y Alejandro Veiga. Facultad de Ingeniería Universidad Nacional de La Plata. Julio de 2025.

{\bf Tesista: Valentín Catácora} dirigido por Dr. Ing. Enrique Spinelli y Mariano Fernández-Corazza. Facultad de Ingeniería Universidad Nacional de La Plata. Julio de 2023.

{\bf Tesista: Humberto M. Celleri} dirigido por Dr. Ing. Martín Sánchez. Doctorado en Ingeniería, Mención Materiales UTN-FRLP. Diciembre de 2021. 

{\bf Tesista: Patricio Gross} dirigido por Dr. Ing. Juan Bava. Facultad de Ingeniería Universidad Nacional de La Plata. Diciembre de 2019. 
 
{\bf Tesista: Ana González Suárez} dirigida por Dr. Ing. Enrique Berjano. Universidad Politécnica de Valencia. 24 de febrero de 2014.

\subsection{\sc 3.6 Integrante de tribunal de proyecto final de carrera}

{\bf Jurado de Prácticas Profesionales Supervisadas (PPS)} Universidad Tecnológica Nacional Facultad Regional La Plata, años 2022 al presente.

{\bf Jurado de Prácticas Profesionales Supervisadas (PPS)} Universidad Nacional Arturo Jauretche, Diciembre 2023,2024.

{\bf Tesina final de carrera Ingeniería en Alimentos de Agustina Lago} dirigida por: Dr. Andrés G. Salvay y codirigida por: Dra. Mercedes A. Peltzer. Universidad Nacional de Quilmes. Noviembre de 2020.
%
% {\bf Práctica Profesional Supervisada (PPS) de Leandro Jesús Charlier} dirigido por Dr. Marcelo Cappelletti y Dr. Waldo Hasperué. Universidad Nacional Arturo Jauretche. Julio de 2017.

\section{\sc 4 Formación de recursos humanos en Investigación, Desarrollo e Innovación}
\vspace*{-.2in}
\subsection{\sc 4.1 Dirección o codirección de tesis de doctorado}

{\bf Codirección de Dr. Jesús E. Fajardo}, dirigido por Dr. Fernando Vericat. Inscripto en 2015 en el doctorado de la Facultad de Ciencias Exactas, Universidad Nacional de La Plata. {\bf Finalizado} en febrero de 2020.

{\bf Codirección de Ing. Christian Botta}, dirigido por Dr. Marcelo Cappelletti. Inscripto en 2024 en el doctorado en Ciencia y Tecnología de la Universidad Nacional de Quilmes.

{\bf Codirección de Ing. Lucas O. Basiuk}, dirigido por Dr. Manuel Carlevaro. Inscripto en 2022 en el doctorado en Ingeniería Mención Materiales, UTN-FRLP con beca CONICET. Fecha de finalización estimada: 2026.

{\bf Dirección de Ing. María José Cervantes}. Inscripta en abril de 2021 en el doctorado de la Facultad de Ciencias Exactas, Universidad Nacional de La Plata con beca CONICET. Fecha de finalización estimada: 2025.

\subsection{\sc 4.2 Dirección de proyecto final de grado}

{\bf Dirección de Matías Benary}. Año de defensa 2024. Ingeniería en Informática, Universidad Nacional Arturo Jauretche.

{\bf Dirección de Diego Luparello}. Año de defensa 2023. Ingeniería en Informática, Universidad Nacional Arturo Jauretche.

{\bf Dirección de Javir Gómez Monteiro}. Año de defensa 2023. Ingeniería en Informática, Universidad Nacional Arturo Jauretche.

{\bf Co-Dirección de Hernán Poncetta}. Año de defensa 2022. Ingeniería Mecánica, Universidad Tecnológica Nacional Facultad Regional La Plata.

{\bf Codirección de Jesús Serrano} dirigido por Profesora Eugenia Blangino. Año de defensa 2011. Departamento de Ingeniería Mecánica, Facultad de Ingeniería, Universidad de Buenos Aires.

\subsection{ \sc 4.3 Dirección o codirección de investigadores}

Catalina Cely Ortíz. Beca posdoctoral de AGENCIA. \textbf{Director de beca de 2024 a la actualidad}.

Hernán Sánchez. Investigador Asistente de CONICET. \textbf{Codirector de carrera de 2022 a la actualidad}.
%
% Hugo Ariel Álvarez. Docente Investigador UNLP. \textbf{Director} (Proyecto UNAJ Investiga 2017, (2018-2019)).
%
% Federico Lotto. Becario Posdoctoral CONICET. \textbf{Director} (Proyecto PICT 2016-2303, (2018-2020)).
%
% Ariel G. Meyra. Investigador Adjunto CONICET. \textbf{Director} (Proyecto PICT 2016-2303, (2018-2020)).

\subsection{\sc 4.4 Dirección de becarios de grado}

{\bf Dirección de beca BIEI 2023}. Estudiante: Diego Luparello,  UNAJ.

{\bf Dirección de beca de Investigación y Servicio 2021 (SAE)}. Estudiante: Hernán Poncetta,  UTN-FRLP.

{\bf Dirección de beca Estímulo a las Vocaciones Científica 2019}. Estudiante: Julián Gómez, Universidad Nacional Arturo Jauretche.

{\bf Dirección de beca Estímulo a las Vocaciones Científica 2018}. Estudiante: Florencia Ayala, Universidad Nacional Arturo Jauretche.
%
% {\bf Dirección de beca Estímulo a las Vocaciones Científica 2015}. Estudiante: Luciano Olmedo, Universidad Nacional Arturo Jauretche.
%
% \subsection{\sc 4.5 Dirección de auxiliares de docencia}
%
% Como {\bf Profesor Titular} de Introducción a los Elementos Finitos en la UTN-FRLP desde 2019 a la fecha tengo a cargo al Profesores Adjunto Ariel G. Meyra.
%
% Durante mis periodos de {\bf Profesor Adjunto y Titular} tanto en FBA-UNLP como en FI-UNLP entre 2009 a 2017 tuve a cargo Profesores Adjuntos, Jefes de Trabajos Prácticos, Ayudantes Diplomados, y Ayudantes Alumnos.

\section{\sc 5 Actividad en investigación, desarrollo e Innovación}
\vspace*{-.2in}

\subsection{\sc 5.1 Director de proyecto homologado}

\textbf{Director de ``Resolución de problemas biomédicos y biomiméticos por elementos finitos''}. Otorgado por la UTN, PID UTN MAECLP0009851TC, (2023--2026)

\textbf{Director de ``Tomografía de microondas: algoritmos de reconstrucción, validación experimental y aplicaciones''}. Otorgado por la Agencia Nacional de Promoción Científica y Tecnológica, PICT-2020-SERIEA-00457, (2022--2024).

\textbf{Director de ``Desarrollo e implementación de una metodología para la evaluación \textit{in vivo} de la calidad ósea''}. Otorgado por la Agencia Nacional de Promoción Científica y Tecnológica, PICT 2016-2303, (2018--2020).

\textbf{Director de ``Resolución computacional de problemas térmicos y electromagnéticos en aplicaciones biomédicas''}. Proyecto UNAJ Investiga 2017 código: 80020170100019UJ, otorgado por la Universidad Nacional Arturo Jauretche, (2019--2020).

\textbf{Director de ``Simulación y evaluación de propiedades dieléctricas de tejidos óseos''}. Proyecto de Investigación Científica y Tecnológica (PICT-2012 Jóvenes). Otorgado por Ministerio de Ciencia, Tecnología e Innovación Productiva / Agencia Nacional de promoción Científica y Tecnológica, (2013--2015).

\textbf{Subsidios para Jovenes Investigadores 2011}. Otorgado por Universidad Nacional de La Plata, (2011--2012).

\subsection{\sc 5.2 Codirector de proyecto homologado}

\textbf{``Soluciones a problemas inversos en ingeniería biomédica basados en representaciones sparse y algoritmos de inteligencia artificial''}. Proyecto de Investigación Plurianual CONICET (PIP código: 11220210100284CO). Participan: Instituto Argentino de Radioastronomía e IFLySiB. Otorgado por CONICET, (2023-2024).

\textbf{``Utilización de métodos numéricos y de técnicas subsimbólicas de la inteligencia artificial para la generación de energías  limpias y la evaluación de la salud ósea''}. Proyecto de Investigación Orientado (PIO-00001). Participan: Universidad Nacional Arturo Jauretche, IFLySiB. Otorgado por CONICET, (2016-2018).

\textbf{``Algoritmos de machine learning para procesamiento de imágenes en aplicaciones biomédicas, agronómicas y ambientales''}. UNAJ Investiga 2023-Modalidad 1-Tipo A. Participan: Universidad Nacional Arturo Jauretche. Otorgado por CONICET, (2023-2025).

\subsection{\sc 5.3 Evaluación de proyectos científicos, tecnológicos e ingeniería}

{\bf Convocatoria PICT-2022 a la actualidad, Co-Coordinador del Área Tecnología Informática, de las Comunicaciones y Electrónica del FONCyT}. Agencia I+D+i, MinCyt.

{\bf Convocatoria PICT-2022}. Evaluador de Proyectos de Investigación Científica y Tecnológica (2022) - integrando el Banco de evaluadores del FONCyT a la fecha.

{\bf Convocatoria Proyectos de Investigación 2021-2022}. Facultad de Ciencias Exactas y Naturales Universidad Nacional de Mar del Plata.

{\bf Convocatoria PICT-2017}. Evaluador de Proyectos de Investigación Científica y Tecnológica (2017) Plan Argentina Innovadora 2020 - Jóvenes, integrando el Banco de evaluadores del FONCyT, 11 de setiembre de 2017.

{\bf Proyecto de investigación Fundamental Fondo Clemente Estable 2017}. Agencia Nacional de Investigación e Innovación (Uruguay).
%
% {\bf Convocatoria PICTO-UNLaM-2013}. Evaluador de proyecto, integrando el Banco de evaluadores del FONCyT, 26 de noviembre de 2014.

\subsection{\sc 5.4 Integrante de proyecto con cuatro años o más de antigüedad}

\textbf{Estudio de fluidos confinados en sistemas de interés tecnológico}. Proyectos de Investigación de Unidades Ejecutoras 2018, en ejecución otorgado por CONICET, Titular: Dr. Santiago Grigera. Responsable científico: Dr. Manuel Carlevaro, (2019--2023). 

\textbf{Termodinámica Estadística}. Otorgado por la Universidad Nacional de La Plata, Proyecto 11/I108. Titular: Fernando Vericat.(2006--2009). 

\subsection{\sc 5.5 Integrante de proyecto}

\textbf{Mejora de las terapias ablativas de vanguardia mediante el control del comportamiento tisular y celular usando campos electromagnéticos}. Participan: Universidad Politécnica de Valencia, Estudios en Neurociencias y Sistemas Complejos (CONICET), Instituto de Investigaciones en Electrónica, Control y Procesamiento de Señales (LEICI CONICET), e IFLySiB CONICET. Otorgado por Ministerio de Economía y Competitividad, España, (2015--2017).

\textbf{Flujo y transporte de material granular en sistemas de interés tecnológico}. Grupo de Materiales Granulares. Otorgado por la UTN, PID UTN MAUTILP0007746TC (2020--2023).

\textbf{Tecnologías basadas en energía de radiofrecuencia y microondas para cirugía de mínima invasión}. Participan: Universidad Politécnica de Valencia, IFLySiB. Otorgado por Ministerio de Economía y Competitividad, España, (2015--2017).

\textbf{Líquidos clásicos y fermiónicos: Estudio teórico y computacional}. Proyecto plurianual otorgado por el CONICET, PIP 112-201201-00154. Titular: Fernando Vericat. (2014--2016).

\textbf{Termodinámica estadística}. Otorgado por la Universidad Nacional de La Plata. Proyecto 11/I153. Titular: Fernando Vericat. (2010--2012). 

\textbf{Mise en place d’un laboratoire international pour la recherche sur les matériaux structures recouvrements pour applications médicales}. Participan: Université Laval-INTI-Laboratorio de Biomecánica FI-UBA. Otorgado por Agence Universitaire de la Francofonie (AUF), (2010--2012).

\textbf{Estudio teórico y computacional de líquidos.} Proyecto plurianual otorgado por el CONICET, PIP  112-200801-01192. Titular: Fernando Vericat. (2009--2011).

\textbf{Propiedades termodinámicas, estructurales y electrónicas de líquidos. Teoría y simulación.} Otorgado por la Agencia Nacional de Promoción Científica y Tecnológica, PICT 2007-00908. Titular: Fernando Vericat. (2008--2009).

\textbf{Teoría y Simulación de Líquidos.} Proyecto plurianual otorgado por el CONICET, PIP Nro. 6240. Titular: Fernando Vericat. (2006--2008).

\bigskip

\section{\sc 6 Producción en investigación, desarrollo, innovación y transferencia}
\vspace*{-.2in}

\subsection{\sc 6.1 Capítulos de libros} 

Capítulo de libro \textbf{Radiofrequency ablation} en ``Principles and Technologies for Electromagnetic Energy Based Therapies'', autores: M. Trujillo, A. González-Suárez, R. M. Irastorza, J. J. Pérez, E. Berjano, publicado en 2022.


\subsection{\sc 6.2 Publicaciones con referato en revistas internacionales}

Berjano E., {\bf Irastorza R. M.}, ``Positioning of the dispersive electrode and its effect on the safety and efficacy of radiofrequency ablation'', EP Europace, (2025).

Collavini S., Pérez J.J., Berjano E., Fernández-Corazza M., Oddo S., {Irastorza R.M.},
``Impact of surrounding tissue-type and peri-electrode gap in stereoelectroencephalography guided (SEEG) radiofrequency thermocoagulation (RF-TC): a computational study'', International Journal of Hyperthermia, (2024).

{\bf Irastorza R.M.}, Hadid C., Berjano E., ``Effect of dispersive electrode position (anterior vs. posterior) in epicardial radiofrequency ablation of ventricular wall: A computer simulation study'', International Journal for Numerical Methods in Biomedical Engineering, (2024).

Cervantes M. J., Basiuk L. O., González-Suárez A., Carlevaro C. M., \textbf{Irastorza, R.M.}, ``Low-Frequency Electrical Conductivity of Trabecular Bone: Insights from In Silico Modeling'', Mathematics, (2023).

\textbf{Irastorza R.M.}, Maher T., Barkagan M., Liubasuskas R., Berjano E., d’Avila A, ``Anterior vs. posterior position of dispersive patch during radiofrequency catheter ablation: insights from in silico modelling'', EP Europace (2023).

Moll X., Fondevila D., García-Arnas F., Burdio F., Trujillo M., \textbf{Irastorza R.M.}, y otros, ``Comparison of two radiofrequency-based hemostatic devices: saline-linked bipolar vs. cooled-electrode monopolar'', International Journal of Hyperthermia (2022).

\textbf{Irastorza R.M.}, Maher T., Barkagan M., Liubasuskas R., Pérez J.J., Berjano E., y otros, ``Limitations of baseline impedance, impedance drop and current for radiofrequency catheter ablation monitoring: insights from in silico modeling'', Journal of Cardiovascular Development and Disease (2022).

González-Suárez A., \textbf{Irastorza R.M.}, Deane S., O'Brien B., O'Halloran M., y otros, ``Full torso and limited-domain computer models for epicardial pulsed electric field ablation'', Computer Methods and Programs in Biomedicine (2022).

Aryana A., \textbf{Irastorza R.M.}, Berjano E., Cohen R.J., Kraus J., Haghighi‐Mood A., y otros, ``Radiofrequency ablation using a novel insulated‐tip ablation catheter can create uniform lesions comparable in size to conventional irrigated ablation catheters while using a'', Journal of Cardiovascular Electrophysiology (2022).

González-Suárez A., Pérez J.J., \textbf{Irastorza R.M.}, D'Avila A., Berjano E., ``Computer modeling of radiofrequency cardiac ablation: 30 years of bioengineering research'', Computer Methods and Programs in Biomedicine (2022).

\textbf{Irastorza R.M.}, Bovaira M., García-Vitoria C., Muñoz V., Berjano E., ``Effect of the relative position of electrode and stellate ganglion during thermal radiofrequency ablation: a simulation study'', International Journal of Hyperthermia (2021).

Madrid M.A., \textbf{Irastorza R.M.}, Meyra A.G., Carlevaro C.M., ``Self-assembly of self-propelled magnetic grains'', EPJ Web of Conferences (2021).

\textbf{Irastorza R.M.}, Gonzalez-Suarez A., Pérez J.J., Berjano E., ``Differences in applied electrical power between full thorax models and limited-domain models for RF cardiac ablation'', International Journal of Hyperthermia (2020).

Fajardo J.E., Galvan J., Vericat F., Irastorza G., Carlevaro C.M., \textbf{Irastorza R.M.}, ``Phaseless Microwave Imaging of dielectric cylinders: an Artificial Neural Networks-Based Approach'', Progress In Electromagnetics Research, (aceptado 2019). 

Fajardo J.E., Lotto F.P., Vericat F., Irastorza G., Carlevaro C.M., \textbf{Irastorza R.M.}, ``Microwave Tomography with phaseless data on the calcaneus by means of artificial neural networks'', Medical \& Biological Engineering \& Computing, (aceptado 2019).

Fajardo J.E., Vericat F., Irastorza G., Carlevaro C.M., \textbf{Irastorza R.M.}, ``Sensitivity analysis on imaging the calcaneus using microwaves'', Biomedical Physics \& Engineering Express (2019).

Sánchez H.R., {\bf Irastorza R.M.}, Carlevaro C.M., ``Uncertainties and temperature correction in molecular dynamic simulations of dielectric properties of condensed polar systems'', Journal of Molecular Liquids (2019).

Fajardo J.E., Vericat F., Carlevaro C.M., Berjano E., \textbf{Irastorza R.M.}, ``Effect of the trabecular bone microstructure on measuring its thermal conductivity: A computer modeling-based study'', Journal of thermal biology (2018).

\textbf{Irastorza R.M.}, d'Avila A., Berjano E., ``Thermal latency adds to lesion depth after application of high‐power short‐duration radiofrequency energy: Results of a computer‐modeling study'', Journal of cardiovascular electrophysiology (2018).

\textbf{Irastorza R.M.}, Trujillo M., Berjano E., ``How coagulation zone size is underestimated in computer modeling of RF ablation by ignoring the cooling phase just after RF power is switched off'', International journal for numerical methods in biomedical engineering (2017).

\textbf{Irastorza R.M.}, Trujillo M., Martel Villagrán J., Berjano E., ``Computer modelling of RF ablation in cortical osteoid osteoma: Assessment of the insulating effect of the reactive zone'', International Journal of Hyperthermia (2016).

Carlevaro C.M., {\bf Irastorza R.M.}, Vericat F., ``Chirality in a quaternionic representation of the genetic code'', Biosystems, (2016).

Carlevaro C. M., \textbf{Irastorza R. M.}, Vericat F., ``Quaternionic representation of the genetic code'', Biosystems (2016).

\textbf{Irastorza R.M.}, Drouin B., Blangino E., Mantovani D., ``Mathematical Modeling of Uniaxial Mechanical Properties of Collagen Gel Scaffolds for Vascular Tissue Engineering'', The Scientific World Journal (2015).

\textbf{Irastorza R. M.}, Blangino E., Carlevaro C. M., Vericat F., ``Modeling of the dielectric properties of trabecular bone samples at microwave frequency. Medical \& biological engineering \& computing, (2014).

\textbf{Irastorza R.M.}, Carlevaro C.M., Pugnaloni L.A., ``Exact predictions from the Edwards ensemble versus realistic simulations of tapped narrow two-dimensional granular columns'', Journal of Statistical Mechanics: Theory and Experiment, (2013).

\textbf{Irastorza R.M.}, Carlevaro C.M., Verica F., ``Is there any information on micro-structure in microwave tomography of bone tissue?'', Medical Engineering \& Physics (2012).

Brusca M.I., {\bf Irastorza R.M.}, Cattoni D.I., Ozu M., Chara O., ``Mechanisms of interaction between Candida albicans and Streptococcus mutans: An experimental and mathematical modelling study'', Acta Odontologica Scandinavica, (2012).

\textbf{Irastorza R.M.}, Achilli M., Amadei M., Blangino E., Drouin B., Mantovani D., ``Ultrasonic setup for testing hydrogels: preliminary experiments on collagen gels.'', Advanced Materials Research, (2012).

\textbf{Irastorza R.M.}, Mayosky M.A., Grigera J.R., Vericat F., ``Dielectric properties of natural and demineralized collagen bone matrix'', IEEE Trans. on Dielectrics and Electrical Insulation, (2011).
%
% Blangino E., {\bf Irastorza R.M.}, Valente S., Vericat F., ``Experimental Techniques to Evaluate In Vitro Trabecular Bone Properties and Emerging Numerical Model'', Materials Science Forum, (2010).
%
% {\bf Irastorza R.M.}, Valente S., Vericat F., Blangino E., ``Dielectric Properties in Fresh Trabecular Bone Tissue from 1MHz to 1000MHz: A Fast and Non Destructive Quality Evaluation Technique'', Materials Science Forum, (2010).
{\bf Irastorza R.M.}, Mayosky M.A., F Vericat F., ``Noninvasive measurement of dielectric properties in layered structure: A system identification approach'', Measurement, (2009).

\subsection{\sc 6.3 Publicaciones con referato en revistas nacionales}

Basiuk L.O., GC Willhuber G.C., Bendersky M., Meyra A.G., \textbf{Irastorza R.M.}, y otros. ``Evaluación de Modelo Mecánico de Cuerpos Vertebrales Tratados con Discoplastía'', Mecánica Computacional (2022).

Fajardo J.E., Vericat F., Carlevaro C.M., Irastorza R.M., ``Effects of Cancellous Bone Dielectric Variability on Microwaves Detection Feasibility. A Simulation Study'', Revista Argentina de Bioingeniería  (2018). Congreso Argentino de Bioingeniería 2017.

Guzmán J.V., Cappelletti M.A., Irastorza R.M., Morales D.M., ``Taller de actualización de recursos y herramientas TIC aplicadas a la enseñanza de la Física en la escuela media'', XI Congreso de Tecnología en Educación y Educación en Tecnología (TE\&ET 2016).

Cappelletti M.A., Irastorza R.M., Morales D.M., ``Estudio de dispositivos y materiales semiconductores bajo condiciones adversas de funcionamiento a través de herramientas TCAD'', XVI Workshop de Investigadores en Ciencias de la Computación, (2014).

Irastorza R.M., Vericat F., Grigera J.R., ``Mediciones dieléctricas no invasivas en medios estratificados utilizando líneas coaxiales abiertas'', XI Reunión de Procesamiento de Información y Control, Río Cuarto, Argentina, (2005).

\subsection{\sc 6.4 Presentaciones en reuniones científicas}

Caiafa C.F., Irastorza R.M., ``A sparse coding approach to inverse problems with application to microwave tomography imaging'', Revista Mexicana de Astronomıa y Astrofısica Serie de Conferencias, (2024).

Cervantes M.J., Gómez J., Luparello D., Morales M., Fajardo J., Galván J., Caiafa C.F., Irastorza R.M., ``A software tool for Microwave Tomography'', Congreso Argentino de Bioingeniería,  Congreso Anual de la Sociedad Argentina de Bioingeniería (SABI 2023), Buenos Aires. Presentación oral.

Basiuk L.O., Camino-Willhuber G., Bendersky M., Meyra A.G., Irastorza R.M., Carlevaro C.M., ``Biomechanical Analysis of Percutaneous Cement Discoplasty Based on Cement Distribution'', Congreso Argentino de Bioingeniería,  Congreso Anual de la Sociedad Argentina de Bioingeniería (SABI 2023), Buenos Aires. Presentación oral.

Cervantes M.J., Orzuza M.N., Caiafa C., Irastorza R.M., ``Coupling media in microwave imaging: dielectric properties and temperature dependence'', Congreso Anual de la Sociedad Argentina de Bioingeniería (SABI 2022), San Juan. Presentación tipo poster.

Basiuk L.O., Carlevaro C.M., Irastorza R.M., ``Electrical conductivity of trabecular bone: a preliminar simulation study'', 21st International Conference on Biomedical Applications of Electrical Impedance Tomography, Galway, Irlanda. Presentación oral.

Irastorza R.M., Gonzalez-Suárez A., Berjano E., ``Analysis of electrical current distribution in the thorax during radiofrequency cardiac ablation: Preliminary results from a 2D computer model'', Congreso Anual de la Sociedad Argentina de Bioingeniería (SABI 2020), Uruguay. Presentación tipo poster.

Fajardo J.E., Ayala F., Carlevaro C.M., Vericat F., Irastorza R.M., ``Microestructura y propiedades térmicas de hueso trabecular: Mediciones y simulaciones computacionales'', Congreso Anual de la Sociedad Argentina de Física (AFA 2018), CABA. Presentación tipo poster.

Fajardo J.E., Carlevaro C.M., Vericat F., Irastorza R.M., ``Medición mínimamente invasiva de conductividad y difusividad térmica de tejido óseo'', Congreso Anual de la Sociedad Argentina de Física (AFA 2017), La Plata. Presentación tipo poster.

Fajardo J.E., Carlevaro C.M., Vericat F., Irastorza R.M., ``Modelo realista de tomografía de microondas en el calcáneo'', Congreso Anual de la Sociedad Argentina de Física (AFA 2016), Tucumán. Presentación oral y poster.

Irastorza R.M., Trujillo M., Villagrán J.M., Berjano E., ``Radiofrequency Ablation of Osteoma Osteoide: A Finite Element Study'', IFMBE Proceedings, VI Latin American Congress on Biomedical Engineering, 2015, Paraná, Argentina. 

Fajardo J.E., Carlevaro C.M., Irastorza R.M., Vericat F., ``Diseño y construcción de prototipo para la medición de conductividad y difusividad térmica de tumores óseos'', Congreso Anual de la Sociedad Argentina de Física (AFA 2015).

Irastorza R.M., Carlevaro C.M., Vericat F., ``Tomografía de microondas en tejido óseo: un estudio de simulación'', Congreso Anual de la Sociedad Argentina de Física (AFA 2014), Tandil. Presentación en poster.

Drouin B., Gagnon R., Lacombe J., Irastorza R. M., Mantovani D., ``Viscoelastic models of Collagen hydrogels scaffolds used in vascular tissue engineering'', The World Congress of Biomechanic, 2014, Boston, USA.

Irastorza R.M, Blangino E., Drouin B., Mantovani D., ``Non-Invasive Ultrasonic Assessment of the Mechanical Properties of Hydrogels'', 9th World Biomaterials Congress, 2012, Chengdu, China.

Blangino E., Irastorza R.M., Valente S., Vericat F., ``Experimental Techniques to Evaluate In Vitro Trabecular Bone Properties and Emerging Numerical Model'', International Conference on Processing \& Manufacturing of Advanced Materials, 2009 Berlin, Germany.
%
% \subsection{\sc 6.5 Publicación en revistas sin referato}
%
% Fajardo J.E., Lotto F.P., Vericat F., Carlevaro C.M., Irastorza R.M., ``Microwave Tomography with phaseless data on the calcaneus by means of artificial neural networks'', arXiv preprint arXiv:1902.07777, (2019).
%
% Fajardo, J.E., Vericat, F., Irastorza, G, Carlevaro, C.M., Irastorza, R.M., ``Sensitivity analysis on imaging the calcaneus using microwaves'', arXiv preprint arXiv:1709.04934, (2017).

\subsection{\sc 6.5 Evaluación de actividades científicas y técnicas}

Desde 2012 Revisor en revistas científicas:\\
\begin{itemize}
 \item International Journal of Hyperthermia
 \item International Journal for Numerical Methods in Biomedical Engineering
 \item Medical Engineering \& Physics
 \item Progress In Electromagnetics Research
\end{itemize}

% \subsection{\sc 6.7 Desarrollo tecnológico verificado fehacientemente}

% Convenio de cooperación institucional entre el Instituto Argentino de Radioastronomía (IAR) y el Instituto de Física de Líquidos y Sistemas Biológicos (IFLySiB) CCT--CONICET (20/09/2018). El proyecto de la colaboración se titula ``Imágenes de Tomografía por Microondas''. Se está construyendo el dispositivo (tomógrafo de microondas) y desarrollando el software financiado por los proyectos mencionados PICT-2020-SERIEA-00457, (2022--2024) y PIP código: 11220210100284CO (2023-2024).
%
% \subsection{\sc 6.8 Divulgación científica y/o tecnológica documentada}
%
% Irastorza, R.M., ``Más vale MAV en compu que cien volando'', artículo escrito para CienciaNet \footnote{CienciaNet es un sitio independiente que publica notas breves de divulgación científica desde 2007 prestando especial atención a producciones de científicos en Argentina \url{http://ciencianet.com.ar}.} (2009).
%
% Irastorza, R.M., ``¿Wi Tricity? Transferencia de potencia sin cables'', artículo escrito para CienciaNet (2007).

\subsection{\sc 6.6 Servicios especiales y asistencia técnica verificado fehacientemente}

Servicio Tecnológico de Alto Nivel (STAN, CONICET), R. M. Irastorza; C. M. Carlevaro; A. Meyra; H. R. Sánchez, ``Modelo de precipitación de sales minerales en agua de flowback en presencia de CO2 a alta presión'', 01/01/2021 a 01/04/2021, Energia-Hidrocarburos.

Servicio Tecnológico de Alto Nivel (STAN, CONICET), R. M. Irastorza; C. M. Carlevaro; A. Meyra; H. R. Sánchez, ``Simulación numérica de cinética de reacción de precipitación de sales minerales a partir de suspensiones acuosas en presencia de CO2 a altas presiones y temperaturas'', 01/10/2022 a 01/12/2022, Energia-Hidrocarburos.


% \section{\sc Participación en subsidios} 
% 
% 
% 
% Propiedades termodinámicas, estructurales y electrónicas de líquidos. Teoría y simulación. Participan: IFLYSIB CONICET-UNLP. Otorga: Ministerio de Ciencia, Tecnología e Innovación Productiva / Agencia Nacional de promoción Científica y Tecnológica, (2009-2011).
% 
% \section{\sc Subsidios obtenidos} 
% 
% 
% % \end{list2}
% 
% \section{\sc Formación de recursos\\ humanos} 
% 
% Tesista de posgrado: Fajardo, E.J. Co-dirección de Tesis de doctoral: ``Propiedades dieléctricas y térmicas de tejidos óseos. Aplicaciones biofísicas y biomédicas.''. Lugar de realización: IFLySiB (2015-actualidad).

\end{resume}
\end{document}




